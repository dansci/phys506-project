\documentclass{article}
\usepackage[margin=2cm]{geometry}
\begin{document}

\section*{Simulation of Events}

The SNO+ detector will be approximated as a 6m sphere of uniform
material.  Events will be generated by randomly picking points uniform
over the detector volume.  (Uniform in $r^3$, uniform in $\cos \theta$
and uniform in $\phi$) Once a point $\vec{x_0}$ is picked, photons
will be generated at times sampled from a scintillation time spectrum
($P \propto \exp(-t/\tau)$ to a first approximation), and given a
random propagation direction $\hat{p}$(sampled uniformly over a unit
sphere; uniform in $\cos \theta$ and uniform in $\phi$).  The
detection location,$\vec{x_d}$, will then be given by the point at
which a vector $\hat{p}$ starting at $\vec{x_0}$ would be projected
into the 6m sphere.  Mathematically,

\documentclass{article}
\begin{document}

\title{PHYS506a Final Project}
\date{\today}
\author{Daniel J. Scislowski}

\maketitle

\begin{abstract}
  This paper documents results from fitters based on maximum
  likelihood of photon arrival times, maximum likelihood of photon
  position distribution, and a combined maximum likelihood fit.
\end{abstract}

\section{Introduction}
\label{sec:intro}

The SNO+ experiment is a neutrino direct detection experiment housed
in SNOLAB at the Creighton mine in Ontario\TODO{ref}.  It consists of
scintillating liquid surrounded by light detectors.  The position and
energy of charged particles in the detector can be inferred by
analysis of the light detected by the detectors.  This inference
procedure is known as event reconstruction.

Most reconstruction algorithms rely primarily on the photon arrival
times.  The spatial distribution of detected photons however may
contain useful information for the reconstruction procedure.  This
paper aims to assess the feasibility of extracting this information.

\section{Methods}
\label{sec:methods}

For both of the fitters written, the detector geometry and physics
simulation were simplified as follows. The detector consists of a 6m
radius sphere whose whole surface functions as the detection surface.


\end{document}

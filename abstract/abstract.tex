\documentclass{article}
\usepackage{fancyhdr}
\thispagestyle{fancy}
\lhead{Daniel Scislowski}
\chead{PHYS506 Final Project Abstract}
\rhead{\today}
\setlength{\parindent}{0in}
\begin{document}

Reconstruction algorithms for events in SNO+ generally use the photon
time-of-arrival information to fit for position, and the total
detected photon count to fit for energy.  I will write a maximum
likelihood fitter based on a simplified scintillation model and a
simplified detector geometry which does a combined energy-position
fit.  I will compare the reconstructed vertices to those using
separate fitters to see if any improvements are possible.  If so, and
time permitting, I will extend the geometry to be closer to the actual
SNO+ geometry.


\end{document}

